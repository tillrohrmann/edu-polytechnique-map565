\documentclass[a4paper, 12pt, titlepage]{article}

% Including needed packages
\usepackage[margin=2cm]{geometry}
\usepackage{amsmath}
\usepackage{amssymb}

\title
{{\em MAP 565 - Time series analysis}\\
Mini-project - Smoothing trajectories using Kalman filter\\
{\bf Report}}
\author{KLINTEFELT-COLLET Philippe \and ROHRMANN Till}
\date{}

\begin{document}

\maketitle

\section{Question}

Given the position $P(t)$, velocity $V(t)$ and acceleration $A(t)$ of an object moving along a one dimensional axis. 
Furthermore, given the time discretization by $t_k=\delta k$ for $k\in \mathbb{Z}$.
The discretization quantities are denoted by $P_k=P(t_k)$, $V_k=V(t_k)$ and $A_k=A(t_k)$.
Assuming that $A(t)\approx A(t_{k-1})$ for $t\in (t_{k-1},t_k)$ for all $k\in\mathbb{Z}$ and that the discretization step $\delta$ is small.
Show that the following equations hold:
\begin{eqnarray}
	V_k &\approx& V_{k-1} + \delta A_{k-1}\\
	P_k &\approx& P_{k-1} + \delta V_{k-1} + \frac{\delta^2}{2} A_{k-1}
\end{eqnarray}

\paragraph{Proof}

Since $A(t)$ is the derivative of $V(t)$ it holds.
\begin{eqnarray}
	V(t_k) - V(t_{k-1}) &=& \int_{t_{k-1}}^{t_k} A(t)\ dt
\end{eqnarray}

By using the assumption of $A(t)\approx A(t_{k-1})$ for $t\in (t_{k-1},t_k)$ for all $k\in\mathbb{Z}$ we can transform the integral.
\begin{eqnarray}
	V_k &\approx& V_{k-1} + A_{k-1} \int_{t_{k-1}}^{t_k}\ dt \\
	&=& V_{k-1}+\delta A_{k-1}
\end{eqnarray}

The same reasoning holds for the position. 
Hence we obtain the following derivation.

\begin{eqnarray}
	P(t_k) - P(t_{k-1}) &=& \int_{t_{k-1}}^{t_k} V(t)\ dt\\
	P_k &=& P_{k-1} + \int_{t_{k-1}}^{t_k}\left( V_{t_{k-1}} + \int_{t_{k-1}}^{t} A(t^{\prime})\ dt^{\prime}\right)\ dt\\
	P_k &\approx& P_{k-1} + \delta V_{t_{k-1}} + \int_{t_{k-1}}^{t_k} A_{k-1}(t-t_{k-1})\ dt\\
	&=& P_{k-1} + \delta V_{t_{k-1}} + \frac{\delta^2}{2} A_{k-1}	
\end{eqnarray}

\section{Question}

The state vector is defined as following: $\pmb{X}_t=(P_{k,1},P_{k,2},A_{k,1},A_{k,2})^T$.
Assuming 
\begin{eqnarray}
	\pmb{X}_{k} &=& \Phi \pmb{X}_{k-1} + \Pi \pmb{A}_{k-1} \label{eq}\\
	\Phi &=& \left(
		\begin{array}{cccc}
			1 &0 & \delta & 0\\
			0 & 1& 0 & \delta\\
			0 & 0& 1& 0\\
			0 & 0& 0& 1
		\end{array}
	\right)\\
	\Pi &=& \left(
		\begin{array}{cc}
			\delta^2/2 & 0\\
			0 & \delta^2/2\\
			\delta& 0\\
			0 & \delta
		\end{array}
	\right)\\
	\pmb{A}_{k-1} &=& \left (
		\begin{array}{c}
			A_{k-1,1}\\
			A_{k-1,2}
		\end{array}
	\right)
\end{eqnarray}

\paragraph{What are the conditions that the state-space can be modelled like that?}
This modelling is only correct, if the velocity $V(t)$ equals the acceleration $A(t)$. 
This implies $V(t)=A(t)=\frac{dV}{dt}$ and consequently $V(t)=A(t)=e^t$.

\section{Question}

Assuming the state-space vector now as $\pmb{X}_t=(P_{k,1},P_{k,2},V_{k,1},V_{k,2})^T$.
Furthermore, assume the sequences $(A_{t,1})$ and $A_{t,2}$ are uncorrelated white noise with variance $\sigma^2$.
The equation (\ref{eq}) can now be transformed into:
\begin{eqnarray}
	\pmb{X}_{k} &=& \Phi \pmb{X}_{k-1} + \pmb{W}_{k-1}\\
	\pmb{W}_{k} &\sim& \mathcal{N}(0,Q)\\
	Q&=& \left(
		\begin{array}{cccc}
		\delta^4/4& 0& \delta^3/2 & 0\\
		0 & \delta^4/4 & 0& \delta^3/2\\
		\delta^3/2 & 0& \delta^2 &0\\
		0 & \delta^3/2 & 0 & \delta^2
		\end{array}
	\right)\sigma^2
\end{eqnarray}

\section{Question}

The implementation can be found in the file kalman_filter.m.

\section{Question}

Testing

\end{document}
